\documentclass[10pt, fleqn, aspectratio=1610,usenames,dvipsnames]{beamer}
% Preamble for `beamer' slides
% -- Preamble --
\usepackage{xcolor}
\usepackage{colortbl}
\usepackage{multicol}
\usepackage{multirow}
\usepackage{starfont}
\usepackage{hyperref}
\usepackage{etoolbox}	% for TOC spacing fix
\usepackage{parskip}
\usepackage[framemethod=tikz]{mdframed}

% reduce spacing between TOC entries
\makeatletter
\patchcmd{\beamer@sectionintoc}
	{\vskip1.5em}{\vskip0.5em}{}{}
\makeatother

% template settings
\usetheme[progressbar=frametitle]{metropolis}
\setbeamertemplate{frame numbering}[fraction]
\useoutertheme{metropolis}
\useinnertheme{metropolis}
\usefonttheme{metropolis}
\usecolortheme{spruce}

% TOC font size
%\setbeamerfont{subsubsection in toc}{size=\small}

\setbeamercolor{background canvas}{bg=white}
\metroset{block=fill}  % formats `block' display
\hypersetup{
    colorlinks=true,
    linkcolor=blue}
    
% left margin width
\setbeamersize
{text margin left=0.5cm, text margin right=0.5cm}

% alignment 
\defbeamertemplate{description item}{align left}{\insertdescriptionitem\hfill}
\setbeamertemplate{description item}[align left]

% customize default block display
\setbeamertemplate{blocks}[rounded][shadow=true]
\setbeamercolor{block body}{bg=mLightBrown!08}

% -- Macros
\newcommand{\fsc}[1]{\normalfont\scshape{#1}}
\newcommand{\ul}{\rule[0.1in]{\textwidth}{0.2mm}\\ \vspace{-8pt}}
\newcommand*\red{\color{red}}
\newcommand*\dgreen{\color{OliveGreen}}

% -- Variable values for title page
\author{}  % will mess up template if commented out
\date{}     % omit or add your your own date, default is `today'

% -- Override errors
\vfuzz=30pt		% suppress Overfull vbox in Outline
\hypersetup{
	pdfpagelayout=SinglePage,
	pdfauthor={janegca},
	pdfsubject={Hellenistic, Medieval Astrology},
    pdfkeywords={astrology, Greek astrology, Hellenistic astrology, Medieval astrology},
    pdfcreator={TeXworks with pdfLatex, beamer}
}
% -- The Title Slide --
\title{Reception: What will or will not be}
\subtitle{Notes and examples on Reception.}
	\institute{ \large These notes are based primarily on Masha'allah's \textsl{On Reception} trans. Robert Hand, ARHAT, 2nd ed. 1998, and \textsl{Six Astrological Treatises by Masha'allah:The Book of Reception} trans. James Holden, AFA, 2009.
	
\vspace{1em}
	
	Masha'allah tells us that we learn, from reception, \textsl{"Whether something will be or not; when it will be if it should come to pass; when it will become apparent that it will not be if it should not come to pass; what prohibits the matter in case it should not be; and through whom, and from whence it should be if it should come to pass."}[RH p1]}

% -- The Presentation Slides --
\begin{document}
% slides are contained in frames
\begin{frame}
\centering
\begin{minipage}{0.75\textwidth}
\titlepage
\end{minipage}
\end{frame}

% table of contents
\AtBeginSection[]{
	\begin{frame}[t, allowframebreaks]{Outline}
		%\fontsize{7pt}{5pt}\selectfont
		\small
   		\tableofcontents[
   	  		currentsection,
   	  		sectionstyle=show/show,
   	  		subsectionstyle=show/show/hide,
   	  		subsubsectionstyle=show/show/show/hide] 
	\end{frame}
}
% -------------------------------------------------------------------------
\section{Reception Basics}
\subsection{Chart Elements Examined}
\begin{frame}[t]{Chart Elements}

The following elements of a chart are examined in judging what will or will not be:
\begin{itemize}
\item the 7 planets
\item the 12 zodiac signs
\item the places of  the Exaltations and Falls of the planets
\item the applications and separations of the standard Ptolemaic aspects (\Sextile, \Square, \Trine, \Opposition) and the \Conjunction
\item Mutual receptions
\item return of reception
\item planets giving (pushing) their dispositions
\end{itemize}
\vspace{0.5cm}

The state of the significator's \textsl{disposition} either effects (brings about) or prohibits whatever matters the significator represents.
\end{frame}

\subsection{What Constitutes Reception?}
\begin{frame}[t]{What Constitutes Reception?}
\small
\begin{block}{}
\textsl{Reception} happens when one planet (A) is in the domicile or exaltation of another planet (B) whom it aspects or conjuncts. The aspect (or conjunction) from planet \textsl{A} must \textsl{perfect} (become exact)  before planet \textsl{B} moves into another sign or a third planet (C) intercedes by completing its own aspect or conjunction with planet \textsl{B}.
\end{block}

\vspace{0.1cm}
\begin{columns}[T, onlytextwidth]
\column{0.25\textwidth}
\Mars\ 10 \Aries\ $\Rightarrow$ \Conjunction\ \Saturn\ 15 \Aries \\
\vspace{1.5cm}
\Mars\ 10 \Capricorn\ $\Rightarrow$ \Square\ \Saturn\ 20 \Aries

\column{0.01\textwidth}
\rule{.1mm}{.4\textheight}

\column{0.75\textwidth}
\Mars\ is applying to \Conjunction\ \Saturn\ who is in \Mars's domicile (\Aries), therefore\\
\Mars\ \textsl{receives} \Saturn, but, \\
\Mars\ is not in a domicile (\Capricorn, \Aquarius) or Exaltation (\Libra) of \Saturn, so \\
\Saturn\ \textsl{does not receive} \Mars\ (this is a weaker reception than if \Saturn\ received \Mars) \\
\vspace{0.1cm}
\ul
\Mars\ is in \Saturn's domicile (\Capricorn) applying to \Square\ \Saturn \\
\Saturn\ is in \Mars's domicile (\Aries), therefore, \\
\Mars\ receives \Saturn\ and \Saturn\ receives \Mars\ and \\
there is \textbf{Mutual Reception} [MR] by \textsl{domicile} (strongest form of reception)
\end{columns}
\vspace{0.2cm}
\end{frame}
% ----------------------------------------------------
\begin{frame}[t]{What Constitutes Reception Continued}
Mutual reception [MR] by domicile is the strongest form of reception.

MR can also happen by Exaltation if the two planets are in each other's signs of exaltation. Masha'allah tells us that if the matter being analyzed has to do with a King, the Exaltation rulers will have more authority over the matter than the domicile rulers. This is considered to be the second strongest form of  reception.

As a general rule, reception is stronger if the planet applied to (usually the heavier planet) receives the applying planet (usually the lighter planet).

Later authors also considered reception by triplicity, term, and face but only considered it to be an \textsl{effective} reception if it involved at least two of these minor dignities i.e. received by triplicity and term or triplicity and face or term and face. Masha'allah does mention these minor dignities in his \textsl{Book of Thoughts and Intentions} but only in relation to determining which planet is stronger than another; he does not use them in any of his reception examples.

\end{frame}
\input{02-applications}
\subsection{Committing Disposition}
\begin{frame}[t]{Committing Disposition}
\small
\begin{block}{}
Any planet (A) joining another planet (B) from \textsl{B}'s domicile or exaltation \textsl{commits its disposition} to \textsl{B}, who, because of reception, will perfect the matter signified by \textsl{A}. A disposition may be \textsl{returned}, possibly in worse condition, if the receiving planet (B) is retrograde, USB, or in its Fall. 
\end{block}

Astrologically speaking, the ruler of the domicile (B) is responsible for \textsl{settling} or \textsl{disposing} of all matters within his domicile. If \textsl{B} is averse (\Semisextile\ or \Quincunx) to its own sign he is blind to the needs of the sign and any planets in that sign and so he cannot meet their needs unless there is some mitigating condition such as planets in signs that are:

\begin{itemize}
\item \textbf{Like-engirding:}  \Aries\ \Scorpio, \Taurus\  \Libra, \Gemini\ \Virgo, \Sagittarius\ \Pisces,  \Capricorn\ \Aquarius\ (ruled by the same planet)

\item \textbf{Equally Ascending:}  \Aries\ \Pisces, \Taurus\ \Aquarius, \Gemini\ \Capricorn, \Cancer\ \Sagittarius, \Leo\ \Scorpio, \Virgo\ \Libra\ (rise in the same number of hours)

\item \textbf{Seeing-Perceiving:} \Aries\ \Libra, \Taurus\ \Virgo, \Gemini\ \Leo, \Scorpio\ \Pisces, \Sagittarius\ \Aquarius\ (equally distant from \Cancer\ - \Capricorn)

\item \textbf{Commanding-Obeying:} \Taurus\ \Pisces, \Gemini\ \Aquarius, \Cancer\ \Capricorn, \Leo\ \Sagittarius, \Virgo\ \Scorpio\ (equally distant from \Aries\ - \Libra)
\end{itemize}
In which cases, \Semisextile's act like a \Conjunction; a \Sextile\ has the power of a \Trine; a \Square's power is doubled; a \Quincunx\ acts like an \Opposition
\end{frame}
% ---------------------------------------------------------------------------------------------------
\begin{frame}[t]{Committing Disposition - An Example}
The nature of the aspect qualifies the nature of the request as a hard one that will be completed with difficulty, delay, or labour (\Square, \Opposition); an easy one, completed without labour or difficulty (\Sextile, \Trine), or a familiar, or congenial one, easily done (\Conjunction).

\textbf{Example:}
\begin{columns}[T, onlytextwidth]
\column{0.25\textwidth}
\Sun\ 10 \Aries\ \Square\ \Mars\ 10 \Capricorn

\column{0.01\textwidth}
\rule{.1mm}{.30\textheight}

\column{0.75\textwidth}
the \Sun\ perfects a \Square\  to \Mars, \\
committing his disposition to \Mars, who  \\
receives the \Sun\ in his domicile (\Aries), and so \\
\textsl{accepts} the \Sun's disposition and will perfect it, but \\
due to the \Square\, \\ 
only after some anxiety and errors i.e. not easily
\end{columns}
\end{frame}
% -------------------------------------------------------------------------------------------------
\subsubsection{Committing Disposition - Another Example}
\begin{frame}[t]{Committing Disposition - Another Example}
\textbf{Example:}\footnotemark[1]
\begin{columns}[T, onlytextwidth]
\column{0.25\textwidth}
\Sun\ 1 \Libra\ $\Rightarrow$ \Opposition\ \Saturn\ 25 \Aries

\column{0.01\textwidth}
\rule{.1mm}{.20\textheight}

\column{0.75\textwidth}
the \Sun\ is in \Libra\ (the exaltation of \Saturn) applying to \Opposition\ \Saturn \\
\Saturn\ is in \Aries\ (the exaltation of the \Sun) therefore \\
there is \textbf{Mutual Reception} by \textsl{exaltation}, and \\
the \Sun\ commits his disposition to \Saturn, who accepts it
\end{columns}
\vspace{0.25cm}
According to Masha'Allah, when there is Mutual Reception, \textsl{"there is peace and the matter will be perfected"}; however, if \Saturn\ leaves \Aries\ before the \Opposition\ can perfect, \textsl{"there will be enmities, contrariness, misunderstandings, denials and \Saturn\ will not receive the \Sun"} and so will not take on the \Sun's disposition.

\footnotetext[1]{The examples used by Masha'Allah so far involve planets in their sign of Fall or Detriment which we will see (in later examples), may result in a commitment being returned or corrupted.}
\end{frame}

\subsection{The Joining of Planets}
\begin{frame}[t]{Reception and the Joining of Planets}
\begin{block}{}
 Planets are \textsl{joined together} by recognized aspects (\Sextile, \Square, \Trine, \Opposition) or bodily \Conjunction's that become \textsl{exact by degree} and happen when at least one of the planets involved is in the other's domicile or exaltation.
 \end{block}
\textbf{Benefics joined to Benefics}\footnotemark[1]

If a benefic (\Venus\ or \Jupiter) joins with and is received by another benefic, the reception increases the good indicated by the planets.

\textbf{Benefics joined to Malefics}

If a benefic (\Venus\ or \Jupiter) joins with and is received by a malefic planet (\Mars\ or \Saturn) then the malefics are at peace and \textsl{"their evil goes away"} unless they are joined by a \Square\ or \Opposition, in which case there will be some labour, and error.

\textbf{Malefics joined to Malefics}

If a malefic (\Mars\ or \Saturn) joins with another malefic with reception, they are made good and their evil and impediment goes away.
\vspace{0.5cm}

\footnotetext[1]{Based on the essential nature of the planets; there are conditions under which a benefic can act as a malefic and vice versa.}
\end{frame}

\section{Horary Rules}
\subsection{Horary Questions in General}
\begin{frame}[t]{Horary Questions in General}

A question must be of great concern or necessity for the querent (the one asking the question); and they should not ask further questions with regard to the matter until the first question has been understood and examined i.e. don't begin the analysis with a series of questions; a clear, concise question is best

An astrologer should not look to answer his own questions as \textsl{"it does not suit a wise man to look for himself"}. 

To examine the question, draw up a chart for the time the question is verbally put to the astrologer; or, if in the form of a letter, the moment when the astrologer understands what is asked; calculate the Asc, the MC, the places of the 7 planets, and note who disposits each of them and which houses they are in.
\vspace{0.25cm}
\begin{mdframed}[backgroundcolor=gray!5, rightmargin=2em, leftmargin=2em]
\small
\textbf{Note:} James Holden thinks the calculation of the MC implies a quadrant system of houses, most likely Alchabitius, but he acknowledges that Masha'allah, in later examples, appears to be using a Sign-House system or Equal House system. Holden gives house cusps for the example charts as they appear in his source text but says they were not original to Masha'allah.\footnote{In Holden's translation of Masha'allah's \textsl{The Book of Thoughts and Intentions} he indicates that either the Alchabitius or Equal House systems were used.}
\end{mdframed}

\end{frame}
\subsection{Choosing Significators}
\begin{frame}[t]{Choosing Significators}
Masha'allah's method for choosing which planets will act as significators for the \textsl{querent} (the person asking the question) and the \textsl{quesited} (question's subject matter) can be reduced to:
\begin{itemize}
\item Does the ruler of the Ascendant (L1) aspect the 1st House?
	\begin{itemize}
		\item Yes? use L1 as the main significator of the querent and the \Moon\ as their partner
		\item No? does L1 aspect another planet that aspects the 1st House or one who does so indirectly by that 2nd planet aspecting a 3rd planet that aspects the 1st House?
			\begin{itemize}
				\item Yes? work through these planets and use the \Moon\ as a their partner
				\item No? repeat the same steps using the \Moon\ instead of L1 and if she is suitable, make her the main significator and L1 her partner
			\end{itemize}
	\end{itemize}
\item If neither L1 nor the \Moon\ are usable, see which of the two is the first to leave its sign and judge the effect from that planet's first joining to another planet after entering its new sign, using the other planet (L1 or \Moon) as their partner;\footnote{Normally the \Moon\ is the first to change signs as she moves faster than the other planets.} noting that matters will be slow and inactive until the sign change occurs.
\end{itemize}

\end{frame}
% ------------------------------------------------------------------------------------------
\begin{frame}[t]{Choosing Significators Continued}
\begin{itemize}
\item The main significator of the quesited will be the ruler of the house signifying the matter i.e.  Wealth, ruler of 2nd House (L2); Siblings, ruler of the 3rd House (L3); Friends, ruler of the 11th House (L11); Children, ruler of the 5th House (L5), etc.
\item Planets in the 1st House and the house signifying the matter at hand have a bearing on the conditions surrounding the matter BUT they do not determine the outcome of the matter; that depends on the chosen significators, their aspects and condition.\footnote{Robert Hand says \textsl{"Rulers of houses in general are more important to outcomes than occupants of houses."}(fn1p9)}
	\begin{itemize}
		\item planets in the affected houses, if they are received by their dispositor, \textsl{"indicate the goodness and worthiness of the thing sought"}, if they are not received by their dispositor (which implies no aspect between the two), then they indicate impediments and the worthlessness of what is sought
	\end{itemize}
\item If the ruler of the Asc or the \Moon\ does not aspect the 1st House they are to be considered \textsl{evil and impeded}
\end{itemize}

\end{frame}
\subsection{Outcomes}
\begin{frame}[t]{Outcomes}
\begin{block}{}
\textbf{Axiom:} In all matters \textsl{"when the disposition comes to a planet which receives the lord of the matter sought for, that matter will be made fortunate, and it will be perfected."}
\end{block}

If the significator of the querent joins and is received by the significator of the matter then the matter will be perfected and complete when the joining is complete. The number of degrees to the completed joining give an indication as to timing; depending on the context, they can represent years, months, weeks, days, or hours. If the significators started off Void of Course, needing to enter a new sign before joining with  another planet, then things will go slowly and matters will be prolonged.

\begin{block}{}
\textbf{Axiom:} \textsl{"After that, the ruler of the house of the thing [quesited] is looked at to see to what planet it commits its own disposition after its effecting...if it is a benefic, they say the thing will be made better. And if it is a malefic, they say that the thing will be subsequently destroyed."} [JH]p22
\end{block}

The exception is if the question is about death.

\end{frame}
\subsection{Timing}
\begin{frame}[t]{Timing  [JH 32-33] [RH 27-29]}
When the joining is by aspect (not conjunction) the number of degrees needed to perfect the aspect gives the number of days.

If it doesn't happen then, it will happen when the two join bodily in a strong place, an angle or a place where the significator of the matter rejoices.

If not then, it will happen when the planet handling the disposition is rising from the \Sun, renewing itself.

If the significators started off Void of Course, needing to enter a new sign before joining with  another planet, then things will go slowly and matters will be prolonged [JH 17, RH 8]

Angles strengthen, hurry, and fortify the matter and, for good or ill, whatever is signified by the ruler of the disposition, it will last. [JH p36]

\end{frame}

\section{Life \& Sickness}
\subsection{Example: About Life or Death}
\begin{frame}[t]{Example: About Life or Death [JH p18-19] [RH p10-11]}
If a person asks whether he will die during the year, look for indications that:\footnote{If L1 is impeded, use the \Moon\ in its place.}
\vspace{0.5cm}
\begin{columns}[T, onlytextwidth]
\column{0.5\textwidth}
\textbf{They will escape death:}\\
\begin{itemize}
\item L1 to a benefic who is not L8 and who does not pass on disposition or return it to another 
\vspace{0.25cm}

\item L1 to a planet that receives him, and if they are ill, they will recover from their illness
\vspace{0.25cm}

\item \Moon\ joined to L8 but L1 strong and unimpeded
\end{itemize}

\column{0.01\textwidth}
\rule{.1mm}{.35\textheight}

\column{0.5\textwidth}
\textbf{Death wins out:}\\
\begin{itemize}
\item L1 joined to L8 or vice versa (do not consider the \Moon) 
\vspace{0.25cm}

\item L1 joined to a malefic or L8 without reception and the \Moon\ evil and impeded
\end{itemize}
\end{columns}
\vspace{0.5cm}
To know if the person will fully recover their  'good fortune and health' examine the planet that signifies escape from death; if it aspects the significator's original position in the chart the person will escape all harm.\footnote{Robert Hand fnp11}
\end{frame}
\subsection{Sickness}
\begin{frame}[t]{Sickness [JH]p19-21, [RH]p11-13 }

\begin{columns}[T, onlytextwidth]
\column{0.5\textwidth}
Mercury (L1) is Void of Course, does not aspect the 1st, and is not first to leave his sign. \\
\vspace{0.2cm}
The \Moon\ is also Void of Course but she is \Trine\ the 1st House and she is the first to enter a new sign so we look to her first with \Mercury\ being made the sharer. \\
\vspace{0.2cm}

\textbf{\Moon\ in \Taurus\ \Trine\ 1st House} $\Rightarrow$ enters \Gemini \\
$\Rightarrow$ \Square\ \Venus\ in \Pisces\ and commits her disposition (no mention of reception as the \Moon\ is now in \Gemini\ where \Venus\ has no dignities) \\
\Venus\ $\Rightarrow$ \Sextile\ \Jupiter\ in \Taurus, MR by domicile \\
\vspace{0.2cm}
As \Jupiter\ cannot join with any other planet\footnotemark[1], he ends the disposition and becomes the final authority over the matter and so determines the final outcome.
\vspace{0.2cm}

\column{0.5\textwidth}
\begin{center}
{\includegraphics[width=0.9\textwidth]{charts/21-chart-sickness}} \\
\vspace{-0.2cm}
\end{center}
\end{columns}
\footnotetext[1]{He can only apply to \Saturn\ and he is already separated from him.}
\end{frame}
% ------------------------------------------------------------
\begin{frame}[t]{Sickness Continued}

After examining the \Moon\ committing her dispostion, Masha'allah then looks at \Mercury\ (L1) as sharing in the matter and finds it confirms what the \Moon\ signified:

\textbf{\Mercury\ in \Aries\ Void of Course} $\Rightarrow$ \Taurus \\
$\Rightarrow$ \Sextile\ \Venus\ \Pisces\ who receives him from \Taurus \\
\Venus\ $\Rightarrow$ \Trine\ \Jupiter\ in \Taurus\, MR by domicile \\
And again, \Jupiter\ is the end of the disposition chain, supporting what the \Moon\ indicated.
\vspace{0.2cm}

\Jupiter, a benefic, as the final arbiter of the disposition promises "health and quiet" and the end of the illness after a prolonged period (due to the VOC's) but we are told that from the time of \Venus\ accepting the \Moon's disposition to her joining with \Jupiter, the person would have gradually strengthened, with the illness leaving him completely once \Venus\ separates from \Jupiter\ by one minute.

\end{frame}
% -------------------------------------------------------------------------
\begin{frame}[t]{Sickness (Alternative Scenarios)}
One odd thing here, how does \Venus\ join with \Jupiter's \Sextile\ at 19 \Pisces\ before it joins with \Mars's \Square\ at 15 \Pisces? Even assuming he took the aspect moieties into account, \Venus\ would have been within moiety of \Mars\ \Square\ at 7.5° \Pisces (15 - (7+8)/2)) which is less than \Jupiter's \Sextile\ moiety of 11 \Pisces (19-(9+7)/2). 

\vspace{0.3cm}
Masha'allah does list the \Mars\ connection as an alternative scenario, saying such a connection would have meant death for the querent as \Mars's is L8 and does not receive \Venus\ in \Pisces. So perhaps, the \Moon\ $\rightarrow$ \Venus\ $\rightarrow$ \Jupiter\ connections are also simply to illustrate a point.

He goes on to warn that the \Sun, if not L1, destroys by combustion if he does not receive the planet, and this also applies to his \Square\ or \Opposition. [JH]p23.\footnotemark[1]

\footnotetext[1]{I've seen this before as a planet being protected from combustion if he's in his own signs; here, they have be in the \Sun's signs (\Leo\ or \Aries) where the \Sun\ will receive them and so protect them.}
\end{frame}
\subsection{A LIfe Question}
\begin{frame}[t]{A Life Question}
\begin{columns}[T, onlytextwidth]
\column{0.5\textwidth}
The \Sun\ (L1) and \Moon\ are both VOC, with the \Moon\ being first to leave her sign so looked to her first with the \Sun\ as sharer.\\
\vspace{0.25cm}
\Moon\ 28 \Aries\ $\Rightarrow$ \Taurus \\
$\Rightarrow$ \Square\ \Mercury\ 2 \Aquarius\ \\
\hspace{1em} (\Moon\ commits her disposition to \Mercury, no reception) \\
\Mercury\ is joined to, and received by, \Saturn\ \\
\hspace{1em}(\Mercury\ is within 2° of a perfect \Sextile, already engaged) \\
therefore \Saturn\ signifies the person's health and good fortune\\
\vspace{0.25cm}
\Sun\ 24 \Aquarius\ $\Rightarrow$ \Pisces \\
$\Rightarrow$ \Sextile\ \Jupiter\ (with reception) \\
so the \Sun\ also signifies health and good fortune \\
\vspace{0.25cm}
Both L1 and the \Moon\ indicate the same for the person's health and fortune.
\column{0.5\textwidth}
\begin{center}
{\includegraphics[width=0.9\textwidth]{charts/22-chart-life}}
\end{center}
\end{columns}
\end{frame}
% -------------------------------------------------------
\begin{frame}[t]{A Life Question Continued}

Wanting to know when the sickness would lessen, Masha'allah says he next looked at the House of Sickness, the 6th, which was ruled by \Saturn, and \Venus\ (11 \Capricorn) was separated from \Saturn\ by 7° and moving to \Trine\ \Mars\ (15 \Taurus) with each receiving the other; \Mars\ receiving \Venus\ in his exaltation, \Venus\ receiving \Mars\ in her domicile.

This \textsl{"signified a lessening of pain and the arrival of health"}

\end{frame}

\section{Substance}
\subsection{Finding Wealth or Not}
\begin{frame}[t]{Finding Wealth or Not [JH 29-31][RH 23-25]}
If the question is about wealth in general, and not specific to some source or person, examine L1 and the \Moon\ to see which is the main significator of the querent and which the sharer and use L2 as the significator of the quesited, wealth.

\begin{block}{}
\textsl{"The connection of the ruler of the Asc or the \Moon\ with the ruler of the thing is the attainment of the thing by itself"}, whether joined to a benefic or malefic, with or without reception, \underline{unless} the significator of the quesited commits its disposition to another planet.\footnotemark[1]
\end{block}

If L2 commits its disposition to:
\begin{itemize}
\item a malefic, with reception, \textsl{"the thing will be perfected"}
\item a benefic, with or without reception, \textsl{"he will find wealth...if the dispositor is in a strong [place] or in the angles"} as angles are useful to the 2nd and hasten things; cadent houses delay things
\end{itemize}

Wealth is also shown if the 2nd holds: a benefic, a dignified malefic, or a malefic without dignity that receives L1.

\footnotetext[1]{Hand says this is true only for the stronger of the two, L1 or \Moon.}
\end{frame}
% -----------------------------------------------
\begin{frame}[t]{Finding Wealth or Not Continued}
If L1 joins with L2 or planets in the 2nd, the querent has been seeking wealth but if L2 joins with L1, the wealth will come about without any seeking on the querent's part and they will receive more than they hoped for.

\textbf{If no L1, \Moon, 2nd House Connections}

If L1 or the \Moon\ do not connect to L2 or a planet in the 2nd, see if one of them connects to a benefic (with or without reception) that is in a strong place or angular; if that benefic does not commit its disposition elsewhere, the querent will find wealth. 

And the same applies if it connects to a malefic with reception but if it is without reception, the matter will be destroyed.

\begin{block}{}
A benefic that commits its disposition to a malefic who does not receive it signifies the matter ruled by the benefic will be harmed when the disposition is complete; with reception, the matter is perfected without harm.
\end{block}






\end{frame}
\subsection{Looking to Borrow}
\begin{frame}[t]{Looking to Borrow}
If a person is looking to borrow from another, L1 and the \Moon\ signify the querent, L2, his wealth while L7 and L8 signify the possible lender and their wealth.

The querent will get what he seeks if:
\begin{itemize}
\item L1 or the \Moon\ are joined to L8 or a planet in the 8th
\item L8 is joined to L1 or a planet in the 1st that is a benefic, a dignified malefic, or a malefic that receives it
\item either L1 or the \Moon\ joined to a malefic with reception or a benefic in a strong place
\end{itemize}

But if the person is looking to borrow from the King, use the 11th (Wealth of the King) instead of the 8th.

\end{frame}

\section{Inheritance}
\subsection{Inheritance}
\begin{frame}[t]{Inheritance}
\begin{columns}[T, onlytextwidth]
\column{0.5\textwidth}
L1, \Venus\ (16 \Aquarius), was VOC, separating from \Square\  \Jupiter\ (14 \Taurus), L8 (the deceased person's wealth) with reception
\begin{block}{}
\textsl{"Separation by reception is something foul and a horrible thing" [JH 35]}
\end{block}
\Moon\ (28 \Leo) VOC separating from \Opposition\ \Venus\ (16 \Aquarius) \\
\Moon\ first to leave her sign \Leo\ $\Rightarrow$ \Virgo \\
$\Rightarrow$ \Trine\ \Mars\ (6 \Taurus), neither planet receives the other, and \\
since \Mars\ is not L8 he signifies prohibition \\
\vspace{0.25cm}
\Venus\ moves from \Aquarius\ to \Pisces \\
$\Rightarrow$ \Sextile\ \Mars\ (6 \Taurus), neither planet receives the other \\
shows the same as the \Moon

\column{0.5\textwidth}
\begin{center}
{\includegraphics[width=0.9\textwidth]{charts/40-chart-inheritance}} \\
\small
House cusps are not original, Holden shows \Mercury\ as 20 \Pisces; Hand uses the degrees shown but puts \Mercury\ in the 10th
\end{center}
\end{columns}
\end{frame}

\section{Kingdom}
\subsection{Kingdom}
\begin{frame}[t]{Seeking a Kingdom}
\begin{columns}[T, onlytextwidth]
\column{0.5\textwidth}
\Mercury\Retrograde (L1) $\Rightarrow$ \Conjunction\Sun\ $\Rightarrow$ \Sextile\ \Saturn\ (L10) \\
however \\
\Mars\ $\Rightarrow$ \Square\ \Saturn\ (L10) perfects first \\
cutting off the \Sun\ and denying the querent the kingdom \\
\vspace{0.25cm}
And Masha'Allah said the querent would know his hope was lost when \Mars\ perfected its \Square\ to \Saturn \\
\vspace{0.25cm}

Note that the \Moon\ shows the same thing as she first applies to \Mercury, committing her disposition to him and so he hands both his own and her disposition to the the \Sun. And
while  \Mercury\ receives the \Sun, the \Sun\ is not L10 so the matter at hand is not perfected. \\


\column{0.5\textwidth}
\begin{center}
{\includegraphics[width=0.9\textwidth]{charts/50-chart-kingdom}} \\
\end{center}
\end{columns}
\end{frame}
% -------------------------------------------------------------------
\begin{frame}[t]{Another Question on a Kingdom}
\begin{columns}[T, onlytextwidth]
\column{0.5\textwidth}
\Moon\ 4 \Libra\ (L1) $\Rightarrow$ \Opposition\ \Saturn\ 11 \Aries\ in the 10th \textsl{"completes"} the matter as \Saturn\ receives the \Moon\ in his exaltation. \\
\vspace{0.5cm}
The man had little joy from his advancement; however, because \Saturn\ was in a place where he had no dignity, being in his Fall and furthermore, in a pitted degree and because it hated its own place it gave something hateful but it made what it gave fixed, strong, and stable because it was in an angle.
 
\column{0.5\textwidth}
\begin{center}
{\includegraphics[width=0.9\textwidth]{charts/50-chart-kingdom-1}} \\
\end{center}
\end{columns}
\end{frame}
\input{51-another-kingdom}
\subsection{Will I get the dukedom promised by the King?}
\begin{frame}[t]{Will I get the dukedom promised by the King? [JH p57] [RH p59]}
\begin{columns}[T, onlytextwidth]
\column{0.5\textwidth}
\Jupiter\ is L1, direct, in the 10th (most elevated planet) \\
\hspace{1em}\Sun\ \& \Venus\ $\Rightarrow$ \Square\ from \Sagittarius\ in 1st (received) \\
\hspace{1em}he aspects both his domiciles \Sagittarius\ and \Pisces\ \\
\hspace{1em}he aspects the 7th, connecting to all 4 angles \\
\hspace{1em}signifying he will get his dukedom\\
\vspace{0.5em}
\Mercury\ is L7, retro, a rebel opponent in the matter \\
\hspace{1em}cadent in the 12th \\
\hspace{1em}$\Rightarrow$ \Opposition\ \Saturn\ without reception, and \\
\vspace{0.2cm}
\Saturn\ is retro, cadent, with his dispositor, \Venus, combust; worsening matters, and indicating destruction for the opponent \\
\vspace{0.5em}
\Mercury\ retro, by transit, $\Rightarrow$ \Sextile\ \Jupiter\ with reception indicates the rebel opponent will end by seeking "peace and accommodation" from the querent

\column{0.5\textwidth}
\begin{center}
{\includegraphics[width=0.9\textwidth]{charts/52-chart-dukedom}} \\
\scriptsize
The MC was not given, only the Asc degree. The other cusps are in the text Holden used but are not original to Masha'Allah.
\end{center}
\end{columns}

\end{frame}
% --------------------------------------------------------------
\begin{frame}[t]{Dukedom Chart as Battle Chart}
\footnotesize
\begin{columns}[T, onlytextwidth]
\column{0.5\textwidth}
Masha'allah goes further into the chart, reading it as a 'battle' or 'war' chart; he begins with the \Moon, using its separating to identify the querent and its application to identify the rebel lord. \\
\vspace{0.25cm}
\Moon\ separating \Trine\ \Jupiter\ (not rec'd) \\
\Moon\ applying \Opposition\ \Mars\ in \Cancer\ (mixed reception) \\
\vspace{0.2cm}
The \Moon\ is the significator of the opponent  in the 2nd (support for querent) \\
\Mars\ retro, in 8th, in his Fall; indicates the opponents penury (8th is 2nd from 7th) \\
\vspace{0.2cm}
Masha'allah says this indicates the rebel lord could not pay his army
so the querent bought them off; ending the battle \\
\vspace{0.25cm}
But, as \Mars\ is the dispositor of \Mercury\ (L7), and it is in a strong reception with the \Moon, the rebel lord will not lose everything and, as already noted, \Mercury\ to the \Sextile\ to \Jupiter\ indicates a peaceful conclusion for all involved.\\
\vspace{0.2cm}
This is the last of Masha'allah's examples.
\column{0.5\textwidth}
\begin{center}
{\includegraphics[width=0.9\textwidth]{charts/52a-chart-dukedom}} \\
\end{center}
\end{columns}
\end{frame}

\end{document}