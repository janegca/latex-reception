\subsection{Applications and Separations by \Conjunction\ or Aspect}

\begin{frame}[t]{Applications and Separations by \Conjunction\ or Aspect}
\begin{block}{}
The faster, lighter planet (A) moves to \textsl{join} (by \Conjunction, \Sextile, \Square, \Trine, \Opposition) a slower, heavier planet (B) and in so doing, \textsl{A} commits his disposition to \textsl{B} and the aspect does not leave off until the two planets are separated.
\end{block}

\begin{mdframed}[backgroundcolor=gray!5, rightmargin=2em, leftmargin=2em]
\textbf{Note:} Robert Hand says \textsl{"there is no separating orb"} ([RH fn4p5]) and so once the two planets are no longer in a partile (exact according to degree) configuration, the matter is done; however, other authors may consider planets to be in aspect as long as one planet is in the light of another.\footnotemark[1]
\end{mdframed}

\begin{block}{}
A planet \textsl{applying to (joining)} another planet indicates \textsl{"what will be"}. 
\end{block}

\begin{block}{}
A planet separating from another planet indicates \textsl{"what has passed away"}.
\end{block}

\footnotetext[1]{A planet's own light extends for 1/2 its orb in front and behind e.g. the \Sun's orb is 30°, its light extends 15° behind and 15° ahead; for \Venus, whose orb is 16°, her light extends 8° behind and 8° ahead, etc. This concept is the basis of the traditional astrologer's use of \href{https://www.skyscript.co.uk/aspects.html\#mo}{moiety of orb}.}

\end{frame}