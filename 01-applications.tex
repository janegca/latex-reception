\subsection{Applications and Separations by \Conjunction\ or Aspect}

\begin{frame}[t]{Applications and Separations by \Conjunction\ or Aspect}
\begin{block}{}
The faster, lighter planet (A) moves to \textsl{join by degree} (complete a \Conjunction, \Sextile, \Square, \Trine, or \Opposition) a slower, heavier planet (B) and in so doing, \textsl{A} commits his disposition to \textsl{B} and the joining does not leave off until the two planets are separated.
\end{block}

\begin{mdframed}[backgroundcolor=gray!5, rightmargin=2em, leftmargin=2em]
\small
\textbf{Note:} Robert Hand says \textsl{"there is no separating orb"} [RH fn4p5] and so once the two planets are no longer in a partile (exact according to degree) configuration, the matter is done; however, other authors may consider planets to be joined as long as one planet is in the light of the other.\footnotemark[1]
\end{mdframed}

A planet in another planet's light is connected to that planet, the state of the connection indicates whether it represents a past, present, or future event.\\
\begin{itemize}
\footnotesize
\item[$\bullet$] A planet \textbf{separating} from another planet indicates what has been done and is \textsl{``passing away.''}\\

\item[$\bullet$] A planet is \textbf{truly joined} to another planet when the two are in an exact conjunction or aspect, indicates what is \textsl{``in the present''}.\\

\item[$\bullet$] A planet \textbf{applying to} (\textsl{joining}) another planet  indicates \textsl{"what will be"}. \\
\end{itemize}

\footnotetext[1]{A planet's own light extends for 1/2 its orb in front and 1/2 its orb behind e.g. the \Sun's orb is 30°, its light extends 15° behind and 15° ahead; for \Mars, whose orb is 16°, his light extends 8° behind and 8° ahead, etc.}

\end{frame}