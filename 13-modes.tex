\subsection{1-3 Perfection, Deterioration, and Connection}
\begin{frame}[t]{1-3 Perfection, Deterioration, and Connection}
Sahl summarizes the ways in which a matter can be perfected or destroyed as follows:\footnotemark[1]

\begin{description}[style=nextline]
\item[1. Perfection or Advance] (\textsl{alichel}) when a planet is in an angular or succedent house

\item[2. Deterioration or Retreat] (\textsl{alicher}) when a planet is cadent

\item[3. Conjunction or Connection] (\textsl{alittisal}) if A is applying to \Conjunction\ B, they are connected until they are separated by 1°, or, if A \Conjunction\ B in the same sign, they are connected until A moves past B (the heavier planet) and B is no longer within A's light (1/2 its orb) \\
\vspace{1em}
\textsl{"Whenever one planet aspects another and within its own light strikes the degree of that other, it is said to be conjoined to it; and if it does not strike it within its own light, it is not said to be conjoined but rather "going to conjunction.""} ([JH p14]) \\

\end{description}

\footnotetext[1]{[JH] p12-25. Bonatti, in his \textsl{Considerations} \#4, gives a similar list.}
\end{frame}
% ---------------------------------------
\subsection{4-5 Separation, Translation}
\begin{frame}[t]{4-5 Separation, Translation}
\begin{description}[style=nextline]
\item[3. Connection Cont'd] A planet at the end of a sign, not joined to another, but striking a planet in the next sign with its light or being within the other planet's light, is conjoined to it even though it is blind (averse) to it\footnotemark[1]. e.g. \Moon\ 27 \Aries\ is conjoined to \Venus\ 2 \Taurus \\
\vspace{1em}
\textbf{Planet's Light Before and Behind:} \Sun: 15°; \Moon: 12°; \Venus\ \& \Mercury: 7°; \Jupiter\ \& \Saturn: 9°; \Mars: 8°. \\

\item[4. Separation] (\textsl{alitiuctraf}) separation occurs when one planet overtakes and passes another either by conjunction or aspect but \textsl{"an aspect is from one sign to aother, while a conjunction is said to be from one degree to another"}. \\

\item[5. Translation] \textsl{anualac} translation of light occurs when a planet A is separating from a planet B and immediately applies to a third planet C; A is said to \textsl{carry the nature} of B to C. 
\end{description}

\footnotetext[1]{This could be an argument for out-of-sign conjunctions, especially given what is said of Mode 4.}
\end{frame}
% -------------------------------------------------------
\subsection{Translation of Light}
\begin{frame}[t]{Sahl's example of Translation of Light}
JH p14-15

\end{frame}
% -------------------------------------------------------
\subsection{6 Collection}
\begin{frame}[t]{6 Collection}
\begin{description}[style=nextline]
\item[6. Collection of Light] (\textsl{algemee}) when 2 planet, not in aspect themselves, both connect to a 3rd, heavier planet, that planet is said to collect their light

\begin{columns}[T, onlytextwidth]
\column{0.5\textwidth}
\textbf{Example:} A question about whether a king would acquire a kingdom \\
\ul
\Venus\ (L1) and \Moon\ (L10) are averse \\
\Jupiter\ is in the 10th, the house of the kingdom \\
both \Venus\ and the \Moon\ are joined to \Jupiter \\
\vspace{0.25cm}
\textsl{"This signified the acquisition of the kingdom through the hands of some judge or bishop or the hands of some chosen man to whom both planets freely gave their assent."} (p15). \\
\vspace{0.25cm}
\textbf{Note:} Another instance of a dignified benefic in the quesited house perfecting the matter without regard to reception.
\column{0.5\textwidth}
\vspace{-0.5cm}
\begin{center}
{\includegraphics[width=0.9\textwidth]{charts/61-collection}} \\
\end{center}
\end{columns}
\end{description}
\end{frame}


