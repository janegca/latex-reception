\subsection{1-3 Perfection, Deterioration, and Connection}
\begin{frame}[t]{Sahl's Modes: 1-3 Perfection, Deterioration, and Connection}
Sahl summarizes the ways in which a matter can be perfected or destroyed as follows:\footnotemark[1]

\begin{description}[style=nextline]
\item[1. Perfection or Advance] (\textsl{alichel}) when a planet is in an angular or succedent house (Ibn Ezra says "it indicates a favourable result").

\item[2. Deterioration or Retreat] (\textsl{alicher}) when a planet is cadent (Ibn Ezra says "the person will give up the matter")

\item[3. Conjunction or Connection] (\textsl{alittisal}) if A is applying to \Conjunction\ B, they are connected until they are separated by 1°, or, if A \Conjunction\ B in the same sign, they are connected until A moves past B (the heavier planet) and B is no longer within A's light (1/2 its orb) \\
\vspace{1em}
\textsl{"Whenever one planet aspects another and within its own light strikes the degree of that other, it is said to be conjoined to it; and if it does not strike it within its own light, it is not said to be conjoined but rather "going to conjunction."} [JH p14]\\

\end{description}

\footnotetext[1]{[JH] p12-25. Bonatti, in his \textsl{Considerations} \#4, gives a similar list.}
\end{frame}
% ---------------------------------------
\subsection{4-5 Separation, Translation}
\begin{frame}[t]{Sahl's Modes: 4-5 Separation, Translation}
\begin{description}[style=nextline]
\item[3. Connection Cont'd] A planet at the end of a sign, not joined to another, but striking a planet in the next sign with its light or being within the other planet's light, is conjoined to it even though it is blind (averse) to it\footnotemark[1]. e.g. \Moon\ 27 \Aries\ is conjoined to \Venus\ 2 \Taurus. They "perfect after despair" if they join in the next sign before another connects to either of them (Ibn Ezra p131). \\
\vspace{1em}
\textbf{Planet's Light Before and Behind:} \Sun: 15°; \Moon: 12°; \Venus\ \& \Mercury: 7°; \Jupiter\ \& \Saturn: 9°; \Mars: 8°. \\

\item[4. Separation] (\textsl{alitiuctraf}) separation occurs when one planet overtakes and passes another either by conjunction or aspect but \textsl{"an aspect is from one sign to another, while a conjunction is said to be from one degree to another"}. \\

\item[5. Translation] \textsl{(anualac)} translation of light occurs when a planet A is separating from a planet B and immediately applies to a third planet C; A is said to \textsl{carry the nature} of B to C. 
\end{description}

\footnotetext[1]{This could be an argument for out-of-sign conjunctions, especially given what is said under \textsl{Separation}.}
\end{frame}
% -------------------------------------------------------
\subsection{Translation of Light}
\begin{frame}[t]{Sahl's example of Translation of Light}
JH p14-15

\end{frame}
% -------------------------------------------------------
\subsection{6. Conjunction (Collection) of Light}
\begin{frame}[t]{Sahl's Modes: 6. Conjunction (Collection) of LIght}
\begin{description}[style=nextline]
\item[6. Collection of Light] (\textsl{algemee}) when 2 planets, not in aspect themselves, both connect to a 3rd, heavier planet, that planet is said to collect their light
\vspace{0.25cm}
\begin{columns}[T, onlytextwidth]
\column{0.5\textwidth}
\textbf{Example:} A question about whether a king would acquire a kingdom \\
\ul
\Venus\ (L1) and \Moon\ (L10) are averse (\Semisextile) \\
\Jupiter\ is in the 10th, the house of the kingdom \\
both \Venus\ and the \Moon\ are joined to \Jupiter \\
\vspace{0.25cm}
\textsl{"This signified the acquisition of the kingdom through the hands of some judge or bishop or the hands of some chosen man to whom both planets freely gave their assent."} (p15). \\
\vspace{0.25cm}

\column{0.5\textwidth}
\vspace{-0.5cm}
\begin{center}
{\includegraphics[width=0.9\textwidth]{charts/61-collection}} \\
\end{center}
\end{columns}
\end{description}
\end{frame}
% ------------------------------------------------
\subsection{7 Prohibition}
\begin{frame}[t]{Sahl's Modes: 7 Prohibition ("cutting", "intervention", "nullification")}
Prohibition (\textsl{almane}) is found in 3 modes:\footnotemark[1]  \\
\begin{columns}[T, onlytextwidth]
\column{0.5\textwidth}
\vspace{0.25cm}
\textbf{(i) Abscission ("cutting") of Light} \\
occurs when the significators are conjoining but between them is a 3rd planet who the 1st significator conjoins before it can connect with the 2nd significator. \\

\vspace{0.25cm}
\textbf{Example:} a question of marriage \\
\ul
\Mercury\ (L1) conjoins \Square\ \Mars\ before it conjoins \Trine\ \Jupiter (L7) \\
so \Mars's cuts off \Mercury's light from \Jupiter, prohibiting the marriage. \\

\vspace{0.15cm}
And Sahl says \textsl{"the destruction of this thing would be from the description of the dowry"} as \Mars\ is in the 8th house (the bride-to-be's property).

	
\column{0.5\textwidth}
\vspace{-0.5cm}
\begin{center}
{\includegraphics[width=0.9\textwidth]{charts/62-abscission}} \\
\end{center}
\end{columns}
\footnotetext[2]{Dykes calls the 3 modes  "cutting", "intervention", and "nullification" (p56-7)}
\end{frame}
% --------------------------------------------------
\begin{frame}[t]{Sahl's Modes: 7. Prohibition: "intervention"}
\begin{columns}[T, onlytextwidth]
\column{0.5\textwidth}
\vspace{0.5cm}
\textbf{(ii) "intervention"} occurs when 2 significators are in the same sign and a 3rd planet, between them, conjuncts the heavier, 2nd significator  \\

\vspace{0.25cm}
\textbf{Example:} a question of marriage \\
\ul
\Moon\ is L1 and significator of the querent \\
\Saturn\ is L7 and significator of the marriage \\

\vspace{0.25cm}
\Mars\ is between the \Moon\ and \Saturn\ and already conjoined with \Saturn\ (being only 2° from it), effectively intervening between the \Moon\ and \Saturn\ and so prohibiting the  marriage
	
\column{0.5\textwidth}
\begin{center}
{\includegraphics[width=0.9\textwidth]{charts/63-intervention}} \\
\end{center}
\end{columns}
\end{frame}
% -------------------------------------------------
\begin{frame}[t]{Sahl's Modes: 7. Prohibition: "nullification"}
\begin{columns}[T, onlytextwidth]
\small
\column{0.5\textwidth}
\vspace{0.5cm}
\textbf{(iii) "nullification"} when 2 significators are in the same sign and a 3rd, lighter planet, passes the 1st significator to complete an aspect to the 2nd significator, that joining is nullified by the conjunction as \textsl{"an aspect does not destroy a conjunction, but a conjunction does destroy an aspect"} [JH p16-17]

\vspace{0.25cm}
\textbf{Example:} a question of marriage \\
\ul
\Moon\ is L1 and significator of the querent \\
\Saturn\ is L7 and signifcator of the marriage \\
\Mars\ is seen to be \Conjunction\ \Saturn\ and so \textsl{"consequently it was cutting off the aspect between the \Moon\ and \Saturn"}, prohibiting their joining and hence the marriage.

\vspace{0.2cm}
\textbf{Note:} Although \Mars\ and \Saturn\ are 13° apart, and not conjoined, \Mars, by transit, will eventually conjunct \Saturn\ and so nullify the earlier \Opposition\ to \Saturn\ from the \Moon.

\column{0.5\textwidth}
\begin{center}
{\includegraphics[width=0.9\textwidth]{charts/64-nullification}} \\
\end{center}
\end{columns}
\end{frame}
% ------------------------------------------------------
\begin{frame}[t]{Sahl's Modes: 7. Prohibition: "nullification" cont'd}
\begin{columns}[T, onlytextwidth]
\column{0.5\textwidth}
\footnotesize
Another example, when two planets are joined in one sign (\Moon\ and \Mars\ here) and the lighter planet (\Moon) joins a third planet (\Venus) by aspect and commits its disposition before it completes its \Conjunction\, the judgment is from the planet that is in the same sign because \textsl{"conjunction of this sort is, as we have said, stronger than an aspect."} \\
\vspace{0.1cm}
Sahl also says \textsl{"And when one planet is joined to another, but before it comes to it, it is joined to a third; and when it has been joined to that one, the conjunction itself is destroyed"} [JH 17]

\vspace{0.25cm}
I think he's speaking of 3 planets (A B C) together with A joined to both B and C, in this case A connects to B first and B handles the matter; A's joining with C is broken.\footnotemark[1] 

\column{0.5\textwidth}
\begin{center}
{\includegraphics[width=0.9\textwidth]{charts/64a-nullification}} \\
\end{center}
\end{columns}
\footnotetext[1]{Unfortunately Dykes text doesn't have this sentence so no way to double check my reading; however, Ibn Ezra describes the same under his definition of Prohibition (p.122)}
\end{frame}
% --------------------------------------------
\subsection{8 Reception}
\begin{frame}[t]{Sahl's Modes: 8. Reception}
\small
\begin{columns}[T, onlytextwidth]
\column{0.5\textwidth}
\textsl{(alchobol)} Reception occurs when a planet joins another from that planet's domicile or exaltation (perfect reception) or from two of the other planet's minor dignities (triplicity, term, face). \\

\vspace{0.25cm}
\textbf{Example:} \\
\ul
\Moon\ in \Aries\ in \Trine\ to \Mars\ in \Sagittarius \\

\vspace{0.25cm}
\Mars\ will receive the \Moon\ because she occupies his (\Mars) domicile. Similarly, if she was joined to the \Sun, as \Aries\ is his exaltation; or if she was in \Taurus\ and joined to \Venus, or in \Gemini\ and joined to \Mercury; in each case she would be received as she would occupy the domicile or exaltation of the planet she was conjoining. \\

\vspace{0.25cm}
And if the \Moon\ was void of course but on moving into the next sign she was joined to the domicile or exaltation ruler of the first sign, she will be received but if she first joins another planet in her new sign, she will be impeded [JH p19]


\column{0.5\textwidth}
\begin{center}
{\includegraphics[width=0.9\textwidth]{charts/65-reception}} \\
\end{center}
\end{columns}
\end{frame}
% ---------------------------------
\subsection{9 Not Received}
\begin{frame}[t]{Sahl's Modes: 9. Not Received}
\footnotesize
\begin{columns}[T, onlytextwidth]
\column{0.5\textwidth}
\textsl{(gattalchobol)} There can be no reception between two planets (A $\Rightarrow$ B) if: \\
\vspace{0.25cm}
(i) A is in the sign of B's Fall (\Moon\ \Square\ \Saturn) \\
(ii) A is in its Fall and B is peregrine there (\Venus\ \Sextile\ \Jupiter) \\
(iii) B is peregrine in the sign A occupies (\Jupiter\ \Trine\ \Saturn)\\
(iv) B is in the sign of A's Fall (\Mercury\ \Square\ \Mars) \\
(v) B is in the sign of its own Fall \\
\hspace{1em}i.e. \Moon\ in \Gemini\ \Trine\ \Sun\ in \Libra\ (not shown) \\
\vspace{0.25cm}
According to Sahl, a planet with no \textsl{testimony} (dignity) in a place cannot recognize the planet that is applying to it and so cannot receive it. \\

And he says A in B's fall \textsl{"will be like someone who comes to him [B] from the house of his enemies--it does not receive it nor esteem it."} (i) \\

And a planet in its own Fall conjoined to a planet with no dignity in that same sign will see the other planet \textsl{"for nothing, as if an unknown garment should be given to anyone asking"}. (ii)\\

And a planet joined to another in its own Fall \textsl{"makes him descend, and it diminishes what will come to him"}. (v)

\column{0.5\textwidth}
\begin{center}
{\includegraphics[width=0.9\textwidth]{charts/66-not-received}} \\
\end{center}
\end{columns}
\end{frame}
% ---------------------------------------
\subsection{10-13 Void of Course (VOC), Return, Giving Virtue}
\begin{frame}[t]{Sahl's Modes: 10-13 Void of Course (VOC), Return}
\begin{description}[style=nextline]
\item[10. Void of Course] \textsl{(halaaceir)} void of course occurs when a planet is separating from one planet and not joined to (in the light of) any other planet before it leaves the sign it is in. (A \textbf{feral} planet forms no aspect, separating or applying, in the sign it occupies.)

\item[11. Return] \textsl{(atrad)} Return occurs when a planet is retrograde or USB. In such situations the planet will return whatever it receives, destroying the matter. The same also occurs if both planets are cadent i.e. \Moon\ in the 6th applying to \Mars\ in the 12th which \textsl{"denotes destruction of the beginning of the question and [also] its end."} And if A is angular and only B is cadent, it signifies the beginning of a matter that will not have an end.

\item[12-13. Giving Virtue and Nature] \textsl{(defaalchota)} a planet in its own domicile or exaltation is joined to another planet i.e. \Moon\ in \Cancer\ or \Taurus\ joined to \Jupiter\ or any other planet \textsl{"gives its virtue"} (nature) and commits its disposition. When she is not in \Taurus\ or \Cancer\ she just gives her dispostion.\footnotemark[1]
\end{description}
\footnotetext[1]{Here I think \textsl{disposition} refers to a planet's significations while \textsl{virtue} refers to its authority and power.}
\end{frame}
% ------------------------------------------------
\subsection{14 Strength of the Planets}
\begin{frame}[t]{Sahl's Modes: 14. Strength of the Planets [to Realize Events]}
\textsl{(alchoe)} There are 11 modes in which a planet can effectively execute the disposition of another planet, these occur when the planet is:
\begin{enumerate}
\item[1.] in a good place i.e. angular or succedent
\item[2.] in its own dignities i.e. domicile, exaltation, or in 2 minor dignities
\item[3.] direct in motion
\item[4.] not joined to a malefic by conjunction, square, or opposition
\item[5.] not in its Fall or joined to a planet in Fall or in a cadent place
\item[6.] received
\item[7.] superior (\Mars, \Jupiter, \Saturn) and oriental of the \Sun\ or inferior (\Mercury, \Venus) and occidental of the \Sun
\item[8.] \textsl{"in their own light"} i.e. a diurnal planet above the horizon in a day chart, a nocturnal planet above the horizon in a night chart
\item[9.] in a fixed sign (\Taurus, \Leo, \Scorpio, \Aquarius)
\end{enumerate}
\end{frame}
% -------------------------------------------------------------------
\begin{frame}[t]{Sahl's Modes: 14. Strength of the Planets Continued}
\begin{enumerate}
\item[10.] in the "heart of the \Sun" which happens when it is in the same degree as the \Sun\footnotemark[1] \textsl{"because then the fortunes increase good fortune and their good, but the malefics both increase and greatly strengthen their evil"}

\item[11.] a masculine planet in masculine quarters (Asc to MC, Desc to IC) and signs or a feminine planet in feminine quarters (MC to Desc, IC to Asc) or signs
\end{enumerate}
\textsl{"These are the testimonies by which the planets are greatly strengthened...for completing the matter when they have received and allowed."} [JH p23] \\ 
or, \\
\textsl{"These are the testimonies which, [when] the planets are in them, they are strong,...when they accept [the management] and make a promise."} [Dykes p67]


\footnotetext[1]{This is usually called \textsl{cazimi} and many authors limit it to a planet within 17' of the \Sun.}
\end{frame}
% --------------------------------------------------------------
\subsection{15 The Debility of the Planets}
\begin{frame}[t]{Sahl's Modes: 15. The Debility of the Planets}
\small
\vspace{-0.2cm}
\textsl{(adoef)} There are 10 modes in which a planet can be debilitated and impeded in a nativity or horary chart; these occur when a planet is:
\vspace{-0.25cm}
\begin{enumerate}
\item[1.] cadent and so not in aspect to the ascending place i.e. in the 6th or 12th place
\item[2.] retrograde
\item[3.] USB i.e. within 17° of the \Sun
\item[4.] \Conjunction, \Square, or \Opposition\ a malefic
\item[5.] besieged by malefics
\item[6.] in its own fall or joined to a planet in its fall
\item[7.] joined to a cadent planet or separating from a planet that was receiving it
\item[8.] peregrine and \textsl{"being pursued by the \Sun, i.e. when it is before the \Sun"}\footnotemark[1]
\item[9.] on the ecliptic (0° latitude) and with the \NorthNode\ or \SouthNode
\item[10.] in its detriment (opposite its domicile) 
\end{enumerate}
\vspace{-0.2cm}
\footnotesize
\textsl{"Follow these modes...with regard to the planet that receives the disposition and the one that promises the matter."} [JH p24]


\footnotetext[1]{Dykes says this means the planet has set before the \Sun\ and is invisible. [p68]}
\end{frame}
% -----------------------------------------------------
\subsection{16 The Defects of the Moon}
\begin{frame}[t]{Sahl's Modes: 16. The Defects of the Moon}
\textsl{(alchamar)} There are 10 ways the \Moon\ can come to be in a bad condition and so detrimental in all questions and initiatives; they are when she is:
\begin{enumerate}
\item[1.] combust i.e within 12° of the \Sun
\item[2.] in her fall (\Capricorn) or joined to a planet in its own fall
\item[3.] within 12° of an exact \Opposition\ with the \Sun
\item[4.] joined to a malefic by \Conjunction, \Square, or \Opposition\ or beseiged by malefics
\item[5.] within 12° of the \NorthNode\ or \SouthNode\ and in the same sign with them
\item[6.] in \Gemini\ (the 12th sign from her domicile) or in the last, and so malefic term, of any sign
\item[7.] cadent or joined to a cadent planet
\item[8.] in the \textsl{Via Combusta}  (19 \Libra\ to 3 \Scorpio)
\item[9.] feral or void of course i.e. not joined to any other planet
\item[10.] slow (moving at less than her average motion) or waning and at the end of the lunar month
\end{enumerate}
\end{frame}
% -------------------------------------------------------------
\begin{frame}[t]{Sahl's Modes: 16. The Defects of the Moon Continued}
\textsl{"These are the...defects of the Moon and its impediments during which no work should be begun,...; and it is not praiseworthy in a nativity [either], or in travel."} [JH p25] \\ \vspace{0.25cm}
\Mars\ harms the \Moon\ most by \Conjunction, \Square, or \Opposition\ when she is waxing as then they are both hot; \Saturn, similarly joined, does not impede her then because he is cold. And in nocturnal charts, when the \Moon\ is waning, and \Mars\ is in a feminine sign, he impedes less but in a day chart, when the \Moon\ is waxing and he is in a masculine sign, he impedes more.

\Saturn\ harms the \Moon\ most by \Conjunction, \Square, or \Opposition\ when she is waning as then they are both cold; \Mars, similarly joined, does not impede her then because he is hot. And in diurnal charts, when the \Moon\ is waxing, and \Saturn\ is in masculine signs, he impedes less but in a nocturnal chart, when the \Moon\ is waning and he is in feminine signs, he impedes more.

\textbf{Impeded:} no planet or sign is impeded unless it is in the \Conjunction, \Square, or \Opposition\ of a malefic.

\textbf{Fortunate:} no planet, or the Asc, is fortunate unless \Venus\ or \Jupiter\ are in its angles.

\end{frame}
% -----------------------------------------------------------------
\begin{frame}[t]{Some of Ibn Ezra's Descriptions of Planet Conditions (Ch.8)}
\footnotesize
\begin{itemize}
\item Any matter will be accomplished according to the strength of the receiving planet
\item A planet is not "declared harmed" unless it is in the light of a malefic
\item A planet separating from aspect of a malefic by at least 1° "will cause fear which is not realized"
\item A benefic aspecting L1, if L1 does not aspect the rising sign, nothing will come to pass
\item Benefics cadent or in signs of opposite nature, detriment, or fall "do not bring benefit at all"; malefics in similar state do little harm
\item Malefics "in great power"  indicate good "except it comes with effort and suffering"
\item \Jupiter\ removes \Saturn's harm (\Venus\ needs \Jupiter's help to do so);  \Venus\ removes \Mars\ harm more so than \Jupiter\ can
\item A planet in Fall "indicates worry, distress, and hardship"
\item A retrograde planet "indicates antagonism, and the destruction of anything that is contemplated"
\item A planet Stationary Direct (SD) about to turn retrograde "is like a person who does not know what to do and the result is unfavourable"; it indicates "difficulty and loss"
\item A planet Stationary Retrograde (SR) about to turn direct "is like a person hoping for something and not losing one's hope"
\end{itemize}
\end{frame}
