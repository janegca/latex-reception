\subsection{16 Modes of Perfection and Destruction}
\begin{frame}[t]{16 Modes of Perfection and Destruction}
Sahl summarizes the ways in which a matter can be perfected or destroyed as follows:\footnotemark[1]

\begin{description}[style=nextline]
\item[1. Perfection or Advance] (\textsl{alichel}) when a planet is in an angular or succedent house

\item[2. Deterioration or Retreat] (\textsl{alicher}) when a planet is cadent

\item[3. Conjunction or Connection] (\textsl{alittisal}) if A is applying to \Conjunction\ B, they are connected until they are separated by 1°, or, if A \Conjunction\ B in the same sign, they are connected until A moves past B (the heavier planet) and B is no longer within A's light (1/2 its orb) \\
\vspace{1em}
\textsl{"Whenever one planet aspects another and within its own light strikes the degree of that other, it is said to be conjoined to it; and if it does not strike it within its own light, it is not said to be conjoined but rather "going to conjunction.""} ([JH p14]) \\

\end{description}

\footnotetext[1]{[JH] p12-25. Bonatti, in his \textsl{Considerations} \#4, gives a similar list.}
\end{frame}
% ---------------------------------------
\begin{frame}[t]{16 Modes Continued}
\begin{description}[style=nextline]
\item[3. Connection Cont'd] A planet at the end of a sign, not joined to another, but striking a planet in the next sign with its light or being within the other planet's light, is conjoined to it even though it is blind (averse) to it\footnotemark[1]. e.g. \Moon\ 27 \Aries\ is conjoined to \Venus\ 2 \Taurus \\
\vspace{1em}
\textbf{Planet's Light Before and Behind:} \Sun: 15°; \Moon: 12°; \Venus\ \& \Mercury: 7°; \Jupiter\ \& \Saturn: 9°; \Mars: 8°. \\

\item[4. Separation] (\textsl{alitiuctraf}) separation occurs when one planet overtakes and passes another either by conjunction or aspect but \textsl{"an aspect is from one sign to aother, while a conjunction is said to be from one degree to another"}. \\
\vspace{1em}
\textsl{Translation of LIght} occurs when a planet A is separating from a planet B and immediately applies to a third planet C, A is said to \textsl{carry the nature} of B to C. (see Sahl's Example Charts)
\end{description}
\footnotetext[1]{This could be an argument for out-of-sign conjunctions, especially given what is said of Mode 4.}
\end{frame}