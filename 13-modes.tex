\subsection{1-3 Perfection, Deterioration, and Connection}
\begin{frame}[t]{1-3 Perfection, Deterioration, and Connection}
Sahl summarizes the ways in which a matter can be perfected or destroyed as follows:\footnotemark[1]

\begin{description}[style=nextline]
\item[1. Perfection or Advance] (\textsl{alichel}) when a planet is in an angular or succedent house

\item[2. Deterioration or Retreat] (\textsl{alicher}) when a planet is cadent

\item[3. Conjunction or Connection] (\textsl{alittisal}) if A is applying to \Conjunction\ B, they are connected until they are separated by 1°, or, if A \Conjunction\ B in the same sign, they are connected until A moves past B (the heavier planet) and B is no longer within A's light (1/2 its orb) \\
\vspace{1em}
\textsl{"Whenever one planet aspects another and within its own light strikes the degree of that other, it is said to be conjoined to it; and if it does not strike it within its own light, it is not said to be conjoined but rather "going to conjunction.""} ([JH p14]) \\

\end{description}

\footnotetext[1]{[JH] p12-25. Bonatti, in his \textsl{Considerations} \#4, gives a similar list.}
\end{frame}
% ---------------------------------------
\subsection{4-5 Separation, Translation}
\begin{frame}[t]{4-5 Separation, Translation}
\begin{description}[style=nextline]
\item[3. Connection Cont'd] A planet at the end of a sign, not joined to another, but striking a planet in the next sign with its light or being within the other planet's light, is conjoined to it even though it is blind (averse) to it\footnotemark[1]. e.g. \Moon\ 27 \Aries\ is conjoined to \Venus\ 2 \Taurus \\
\vspace{1em}
\textbf{Planet's Light Before and Behind:} \Sun: 15°; \Moon: 12°; \Venus\ \& \Mercury: 7°; \Jupiter\ \& \Saturn: 9°; \Mars: 8°. \\

\item[4. Separation] (\textsl{alitiuctraf}) separation occurs when one planet overtakes and passes another either by conjunction or aspect but \textsl{"an aspect is from one sign to aother, while a conjunction is said to be from one degree to another"}. \\

\item[5. Translation] \textsl{anualac} translation of light occurs when a planet A is separating from a planet B and immediately applies to a third planet C; A is said to \textsl{carry the nature} of B to C. 
\end{description}

\footnotetext[1]{This could be an argument for out-of-sign conjunctions, especially given what is said under \textsl{Separation}.}
\end{frame}
% -------------------------------------------------------
\subsection{Translation of Light}
\begin{frame}[t]{Sahl's example of Translation of Light}
JH p14-15

\end{frame}
% -------------------------------------------------------
\subsection{6. Conjunction (Collection) of Light}
\begin{frame}[t]{6. Conjunction (Collection) of LIght}
\begin{description}[style=nextline]
\item[6. Collection of Light] (\textsl{algemee}) when 2 planets, not in aspect themselves, both connect to a 3rd, heavier planet, that planet is said to collect their light

\begin{columns}[T, onlytextwidth]
\column{0.5\textwidth}
\textbf{Example:} A question about whether a king would acquire a kingdom \\
\ul
\Venus\ (L1) and \Moon\ (L10) are averse \\
\Jupiter\ is in the 10th, the house of the kingdom \\
both \Venus\ and the \Moon\ are joined to \Jupiter \\
\vspace{0.25cm}
\textsl{"This signified the acquisition of the kingdom through the hands of some judge or bishop or the hands of some chosen man to whom both planets freely gave their assent."} (p15). \\
\vspace{0.25cm}
\textbf{Note:} Another instance of a dignified benefic in the quesited house perfecting the matter without regard to reception.
\column{0.5\textwidth}
\vspace{-0.5cm}
\begin{center}
{\includegraphics[width=0.9\textwidth]{charts/61-collection}} \\
\end{center}
\end{columns}
\end{description}
\end{frame}
% ------------------------------------------------
\subsection{7 Prohibition}
\begin{frame}[t]{7 Prohibition ("cutting", "intervention", "nullification")}
Prohibition (\textsl{almane}) is found in 3 modes:\footnotemark[1]  \\
\begin{columns}[T, onlytextwidth]
\column{0.5\textwidth}
\vspace{0.5cm}
\textbf{(i) Abscission ("cutting") of Light} \\
occurs when the significators are conjoining but between them is a 3rd planet who conjoins with the 2nd significator before the 1st significator can connect with it.\\

\vspace{0.25cm}
\textbf{Example:} a question of marriage \\
\ul
\Mercury\ (L1) will \Square\ \Mars\ before it can \Trine\ \Jupiter (L7), cutting off \Mercury's light from \Jupiter, prohibiting the marriage. \\

\vspace{0.25cm}
And Sahl says \textsl{"the destruction of this thing would be from the description of the dowry"} as \Mars\ is in the 8th house (the bride-to-be's property).

	
\column{0.5\textwidth}
\vspace{-0.5cm}
\begin{center}
{\includegraphics[width=0.9\textwidth]{charts/62-abscission}} \\
\end{center}
\end{columns}
\footnotetext[2]{Dkykes calls the 3 modes  "cutting", "intervention", and "nullification" (p56-7)}
\end{frame}
% --------------------------------------------------
\begin{frame}[t]{7. Prohibition: "intervention"}
\begin{columns}[T, onlytextwidth]
\column{0.5\textwidth}
\vspace{0.5cm}
\textbf{(ii) "intervention" } when 2 significators are in the same sign and a 3rd planet, between them, conjuncts the heavier, 2nd significator before the 1st significator can reach it \\

\vspace{0.25cm}
\textbf{Example:} a question of marriage \\
\ul
\Moon\ is L1 and significator of the querent \\
\Saturn\ is L7 and significator of the marriage \\

\vspace{0.25cm}
\Mars\ is between the \Moon\ and \Saturn\ and already conjoined with \Saturn\ (being only 2° from it), effectively intervening between the \Moon\ and \Saturn\ and so prohibiting the  marriage
	
\column{0.5\textwidth}
\begin{center}
{\includegraphics[width=0.9\textwidth]{charts/63-intervention}} \\
\end{center}
\end{columns}
\end{frame}
% -------------------------------------------------
\begin{frame}[t]{7. Prohibition: "nullification"}
\begin{columns}[T, onlytextwidth]
\column{0.5\textwidth}
\vspace{0.5cm}
\textbf{(iii) "nullification"} when 2 significators are in the same sign and a 3rd, lighter planet, passes the 1st significator to complete an aspect to the 2nd significator, that joining is nullified by the conjunction as \textsl{"an aspect does not destroy a conjunction, but a conjunction does destroy an aspect"}

\vspace{0.25cm}
\textbf{Example:} a question of marriage \\
\ul
\Moon\ is L1 and significator of the querent \\
\Saturn\ is L7 and signifcator of the marriage \\
\Mars\ is seen to be \Conjunction\ \Saturn\ (event though its fairly wide) and so \textsl{"consequently it was cutting off the aspect between the \Moon\ and \Saturn"}, prohibiting their joining and hence the marriage.

\column{0.5\textwidth}
\begin{center}
{\includegraphics[width=0.9\textwidth]{charts/64-nullification}} \\
\end{center}
\end{columns}
\end{frame}
% ------------------------------------------------------
\begin{frame}[t]{7. Prohibition: "nullification" cont'd}
\small
\begin{columns}[T, onlytextwidth]
\column{0.5\textwidth}
Another example, when two planets are joined in one sign (\Moon\ and \Mars\ here) and the lighter planet (\Moon) joins a third planet (\Venus) and commits its disposition (the two are in mutual reception), before it completes its \Conjunction\, the judgment is from the planet that is in the same sign because \textsl{"conjunction of this sort is, as we have said, stronger than an aspect."} \\

\vspace{0.25cm}
\textbf{Note:} Dykes points out this isn't strictly speaking a case of prohibition as \Venus\ is not one of the significatiors we are interested in. What the example does show is a situation that only \textsl{appears} to be a prohibition. \\

\vspace{0.25cm}
Sahl also says that \textsl{"when one planet is joined to another, but before it connects to it, it is joined to a third; and when it has been joined to that one, the conjunction itself is destroyed"} the implication being that while an aspect cannot destroy a conjunction, another conjunction can.

\column{0.5\textwidth}
\begin{center}
{\includegraphics[width=0.9\textwidth]{charts/64a-nullification}} \\
\end{center}
\end{columns}
\end{frame}
