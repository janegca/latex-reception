\subsection{Committing Disposition}
\begin{frame}[t]{Committing Disposition}
\small
\begin{block}{}
Any planet (A) joining another planet (B) \textsl{commits its disposition} to \textsl{B}. If the disposition is received, the matter signified will be completed. A disposition may be \textsl{returned}, possibly in worse condition, if the receiver is retrograde, USB, or in its Fall.  It may not be lasting if the planets are cadent.
\end{block}

Astrologically speaking, the ruler of a sign (as the receiver of a disposition) is responsible for \textsl{settling} or \textsl{disposing} of all matters signified by his sign. If a ruler is averse (\Semisextile\ or \Quincunx) to its own sign he is blind to the needs of the sign  and so he cannot meet the sign's needs unless there is some mitigating condition i.e. the joined planets are in signs that are:\footnotemark[1]
\vspace{-0.25cm}
\begin{itemize}
\item \textbf{Like-engirding:}  \Aries\ \Scorpio, \Taurus\  \Libra, \Gemini\ \Virgo, \Sagittarius\ \Pisces,  \Capricorn\ \Aquarius\ (ruled by the same planet)

\item \textbf{Equally Ascending:}  \Aries\ \Pisces, \Taurus\ \Aquarius, \Gemini\ \Capricorn, \Cancer\ \Sagittarius, \Leo\ \Scorpio, \Virgo\ \Libra\ (rise in the same number of hours)

\item \textbf{Seeing-Perceiving:} \Aries\ \Libra, \Taurus\ \Virgo, \Gemini\ \Leo, \Scorpio\ \Pisces, \Sagittarius\ \Aquarius\ (equally distant from \Cancer\ - \Capricorn)

\item \textbf{Commanding-Obeying:} \Taurus\ \Pisces, \Gemini\ \Aquarius, \Cancer\ \Capricorn, \Leo\ \Sagittarius, \Virgo\ \Scorpio\ (equally distant from \Aries\ - \Libra)
\end{itemize}
\vspace{-0.25cm}
In which case, a \Semisextile\ acts like a \Conjunction; a \Quincunx\ acts like an \Opposition; and a \Sextile\ has the power of a \Trine; a \Square's power is doubled
\footnotetext[1]{Mitigating conditions are found in \textsl{Late Classical Astrology: Paulus Alexandrinus and Olympiodorus} trans. Dorian Gieseler Greenbaum, ARHAT, 2001 p16-20 and \textsl{Julius Firmicus Maternus: Mathesis} trans. James Holden, AFA, Book II.30.9 p81-82.}

\end{frame}
% ---------------------------------------------------------------------------------------------------
\begin{frame}[t]{Committing Disposition - An Example}
The nature of the aspect qualifies the nature of the request as a hard one that will be completed with difficulty, delay, or labour (\Square, \Opposition); an easy one, completed without labour or difficulty (\Sextile, \Trine), or a familiar, or congenial one, easily done (\Conjunction).

\textbf{Example:}
\begin{columns}[T, onlytextwidth]
\column{0.25\textwidth}
\Sun\ 10 \Aries\ \Square\ \Mars\ 10 \Capricorn

\column{0.01\textwidth}
\rule{.1mm}{.30\textheight}

\column{0.75\textwidth}
the \Sun\ perfects a \Square\  to \Mars, \\
committing his disposition to \Mars, who  \\
\textbf{receives} the \Sun\ in his domicile (\Aries), and so \\
\textsl{accepts} the \Sun's disposition and will perfect it, but \\
due to the \Square\, \\ 
only after some anxiety and errors i.e. not easily
\end{columns}

\end{frame}
% -------------------------------------------------------------------------------------------------
\subsubsection{Committing Disposition - Another Example}
\begin{frame}[t]{Committing Disposition - Another Example}
\textbf{Example:}\footnotemark[1]
\begin{columns}[T, onlytextwidth]
\column{0.25\textwidth}
\Sun\ 1 \Libra\ $\Rightarrow$ \Opposition\ \Saturn\ 25 \Aries

\column{0.01\textwidth}
\rule{.1mm}{.25\textheight}

\column{0.75\textwidth}
the \Sun\ is in \Libra\ (the exaltation of \Saturn) applying to \Opposition\ \Saturn \\
\Saturn\ is in \Aries\ (the exaltation of the \Sun) therefore \\
each planet \textbf{receives} the other in their signs of exaltation \\
so there is \textsl{mutual reception} by \textsl{exaltation}, and \\
the \Sun\ commits his disposition to \Saturn, who accepts it
\end{columns}
\vspace{0.25cm}
According to Masha'allah, when there is mutual reception, \textsl{"there is peace and the matter will be perfected"}; however, if \Saturn\ leaves \Aries\ before the \Opposition\ can perfect, \textsl{"there will be enmities, contrariness, misunderstandings, denials and \Saturn\ will not receive the \Sun"} and so will not take on the \Sun's disposition.

\begin{block}{}
Reception is one key to knowing whether a planet's significations are \textsl{disposed of} i.e. effected and manifested in the life, or not.
\end{block}

\footnotetext[1]{The example used by Masha'allah here involves planets in their signs of Fall which we will see (in later examples), may result in a commitment being returned or corrupted.}
\end{frame}
