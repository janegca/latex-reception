\subsection{Committing Disposition}
\begin{frame}[t]{Committing Disposition}
\small
\textbf{Axiom:} Any planet (A) joining another (B) from B's domicile or exaltation \textsl{commits its disposition} to \textsl{B}, who, because of reception, will perfect the matter signified by \textsl{A}.

The language used here makes sense in that if \textsl{A} is in a domicile of \textsl{B} he is a guest in \textsl{B}'s home; \textsl{B} is \textsl{A's} host, and the one responsible for seeing that his guests get what they need.

In legal terms, a \textsl{disposition} is  \textsl{"a settlement"}.  Astrologically speaking, the ruler of a domicile is responsible for \textsl{settling} or \textsl{disposing} of all matters within his domicile, including those of other planets lodged there.

The nature of the aspect qualifies the nature of the request as a hard one (\Square, \Opposition), an easy one (\Sextile, \Trine), or a familiar, or congenial one (\Conjunction).

\textbf{Example:}
\begin{columns}[T, onlytextwidth]
\column{0.25\textwidth}
\Sun\ 10 \Aries\ $\Rightarrow$ \Square\ \Mars\ 10 \Capricorn

\column{0.01\textwidth}
\rule{.1mm}{.27\textheight}

\column{0.75\textwidth}
the \Sun\ applies to the \Square\ of \Mars, \\
committing his disposition to \Mars, who  \\
receives the \Sun\ in his domicile (\Aries), and so \\
\textsl{accepts} the \Sun's disposition and will perfect it, but \\
due to the \Square\, \\ 
only after some anxiety and errors i.e. not easily
\end{columns}
\end{frame}