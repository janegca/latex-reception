\subsection{Perfection and Outcomes}
\begin{frame}[t]{Perfection and Outcomes}
When a significator of the querent is joined to the significator of the quesited, \textsl{"they say that the thing will be"} [JH p22]

In a general way, \textsl{"When the disposition comes to the planet that receives it, the thing is improved and perfected.... And also, when it comes to a benefic, it perfects by the same disposition. And when its disposition is completed and it has committed it to another [Planet], that dispositor will work through the quantity of its nature, and what there is of it."} [JH p24]

\begin{block}{}
\textsl{Perfection (the performance of a thing) is perhaps known from the planet to which the ruler of the Asc is first joined; from the \Moon; or from the second one that receives the commitment [of the disposition], or namely from whichever one is the last reception [and] there will be the end; or from the ruler of the thing quesited; or from a benefic that is in a good place without reception.} [JH p36]
\end{block}

\end{frame}
% -----------------------------------------
\begin{frame}[t]{Perfection and Outcomes Continued}
\textsl{Perfection} can come about in a number of ways ([RH p31, 34])
\small
\begin{itemize}
\item significator of the querent (A) to the significator of the quesited (B), with or without reception, and \textsl{B} does not commit his disposition to a third planet (C)
\item L1 joining a benefic in the house signifying the matter
\item L1 joining a malefic with dignity in the house signifying the matter
\item L1 joining, and received by, an undignified malefic in the house signifying the matter
\item significator of the quesited to L1, with or without reception, or, to planets in the 1st (under the same conditions described for L1 to planets in the house of the matter signified)
\item L1 or the \Moon\ to an angular or strongly dignified benefic, with or without reception; or, a malefic, similarly placed, with reception
\end{itemize}

\end{frame}
% -----------------------------------------
\begin{frame}[t]{Perfection and Outcomes Continued}

If \textsl{A} is the significator of the querent and receives \textsl{B}, the significator of what is sought; the querent will get what he has been seeking but if \textsl{A} is the significator of what is sought and \textsl{B} is the significator of the querent, then the querent will get what has been asked about without any seeking of it on his part. [JH 38]

Outcomes are better, stronger, more readily done, and stable, and durable if the significator of the quesited receives the significator of the querent. [JH 38]

\begin{block}{}
\textsl{"Reception will not be destroyed in any way whatsoever,...,if the planet which receives has rulership over a matter, and it is the dispositor [of the querent's significator] and the disposition comes to it [the querent's significator applies to it], and if it does not commit disposition to another. Because if it should commit disposition to another planet after its own reception, the reception signifies the accomplishment of the matter, and the commission of its disposition to another planet signifies the end of the matter and to what [state] its final affairs would come."} [RH p32]
\end{block}
So if A applies to its dispositor B, it is received and the matter is accomplished. If B happens to then apply to another planet C, the conditions of that application indicate how the matter will eventually play out.
\end{frame}
% -----------------------------------------
\begin{frame}[t]{Perfection and Outcomes Continued}

\begin{block}{}
\textsl{"After that, the ruler of the house of the thing [quesited] is looked at to see to what planet it commits its own disposition after its effecting...if it is a benefic, they say the thing will be made better. And if it is a malefic, they say that the thing will be subsequently destroyed."} [JH]p22
\end{block}
If an application by L1 or the \Moon\ shows that the matter will be completed; look to the ruler of the matter being sought (the ruler of the quesited) to see how the matter will play out once it has been effected.

The exception is if the question is about death as there is nothing that can come after it.

If the first joining of the querent's significator is to a malefic and does not involve reception and if the malefic is not the significator of the quesited;  a \textbf{prohibition} is indicated [JH p35].

\end{frame}