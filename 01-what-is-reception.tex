\subsection{What Constitutes Reception?}
\begin{frame}[t]{What Constitutes Reception?}
\small
\begin{block}{}
\textsl{Reception} happens when one planet (A) is in the domicile or exaltation of another planet (B) whom it aspects or conjuncts. The aspect (or conjunction) from planet \textsl{A} must \textsl{perfect} (become exact)  before planet \textsl{B} moves into another sign or a third planet (C) intercedes by completing its own aspect or conjunction with planet \textsl{B}.
\end{block}

While Masha'Allah appears to have been the first to begin categorizing and explaining the way reception, applications, and separations indicate outcomes Sahl ibn Bishr al-Israili (786-845) was the first I could find that defined 16 distinct modes of the "perfection and destruction of things". These were also listed and amplified by  Abu Ma'shar (787-886),  ibn Ezra (1089-1164), and much later they are described in Guido Bonatti's \textsl{Liber Astronomaie}.

Examples from all the earlier authors may be used to name and describe the various ways in which reception, or prohibition, can occur. The referenced texts are:
\begin{itemize}
\small
\item \textsl{The Introduction to the Science of the Judgments of the Stars} by Sahl ibn Bishr trans. James Holden, AFA, 2008 
\item \textsl{The Astrology of Sahl B. Bishr Vol. I} trans. Ben Dykes, Cazimi Press, 2019

\item \textsl{The Abbreviation of the Introduction to Astrology} by Abu Ma'shar trans. Charles Burnett, ARHAT, 2nd. printing, 2000

\item \textsl{Ibn Ezra: The Beginning of Wisdom} trans. Meira Epstein, ARHAT, 1998
\end{itemize}

\end{frame}
% -------------------------------------------------------------------
\subsubsection{Perfect Reception}
\begin{frame}[t]{What Constitutes Reception - Perfect Reception}
\vspace{0.1cm}
\begin{columns}[T, onlytextwidth]
\column{0.5\textwidth}
An example of \textbf{perfect reception} (according to Sahl\footnotemark[1]) is a planet applying to the domicile or exaltation ruler of the sign it occupies. Abu Ma'shar calls it \textsl{"pushing nature"}, Ibn Ezra, \textsl{"conferring of nature"}\footnotemark[2] \\
\vspace{0.2cm}
\textbf{Example (A $\Rightarrow$ B): \Moon\ in \Aries\ $\Rightarrow$ \Mars\ or \Sun} 
\ul
\vspace{0.2cm}
\Moon\ in \Aries\ applying to either of \\
\Mars\ (domicile ruler) or \\
\Sun\ (exaltation ruler) \\
\vspace{0.2cm}
\Mars\ or the \Sun\ can be in any sign; they receive the \Moon\ because she is in \Aries\ which \Mars\ rules by domicile and the \Sun\ by exaltation.\\

\column{0.5\textwidth}
\vspace{-0.5cm}

\begin{center}
{\includegraphics[width=0.8\textwidth]{charts/01-perfect-reception}} \\
\end{center}

\end{columns}
\vspace{0.2cm}
\footnotetext[1]{Sahl, p.18}
\footnotetext[2]{Abu Ma'shar p.24 as the sign A occupies has the same nature as the planet B, A pushes B's own nature onto B. Ibn Ezra p.121 says the ruler B confers its nature upon A.}
\end{frame}
% ----------------------------------------------------
\subsubsection{A Weaker Reception}
\begin{frame}[t]{What Constitutes Reception - A Weaker Reception}
\begin{columns}[T, onlytextwidth]
\column{0.5\textwidth}
A planet in its own domicile or exaltation applying to another planet is said by Abu Ma'shar to \textsl{"push [its] power"} to the other planet.\footnotemark[1] Ibn Ezra calls it \textsl{"conferring influence"}, Sahl, \textsl{"giving virtue" or "commiting disposition"}.\footnotemark[2]This too is a form of reception but it is considered weaker than perfect reception.\\
\vspace{0.2cm}
\textbf{Example:} \Mars\ or \Sun\ in \Aries\ $\Rightarrow$  \Saturn \\
\ul
\vspace{0.2cm}
\Mars\ or \Sun\ in \Aries\ applying to  \\
\Saturn\ in any sign \\
\vspace{0.5cm}
Note that Sahl would consider this a 'non-reception' as \Aries\ is \Saturn's Fall.\footnotemark[3]

\column{0.5\textwidth}
\vspace{-0.5cm}
\begin{center}
{\includegraphics[width=\textwidth]{charts/01-weaker-reception}} \\
\end{center}

\end{columns}
\vspace{0.2cm}
\footnotetext[1]{Abu Ma'shar, p.25}
\footnotetext[2]{Ibn Ezra, p. 121, Sahl p20 Holden}
\footnotetext[3]{Sahl, Dykes trans. p.62; Holden's, p.19}
\end{frame}
% -------------------------------------------------
\subsubsection{Pushing Two Natures}
\begin{frame}[t]{What Constitutes Reception - Pushing Two Natures}

\end{frame}

% ----------------------------------------------------
\subsubsection{Mutual Reception}
\begin{frame}[t]{What Constitutes Reception - Mutual Reception}
\begin{columns}[T, onlytextwidth]
\column{0.5\textwidth}
\textbf{Example:} \Mars\ 10 \Capricorn\ $\Rightarrow$ \Square\ \Saturn\ 20 \Aries \\
\ul
\vspace{0.25cm}
\Mars\ is in \Saturn's domicile (\Capricorn) applying to \Saturn \\
\Saturn\ is in \Mars's domicile (\Aries), therefore, \\
\Mars\ receives \Saturn\ by domicile, and \\
\Saturn\ receives \Mars\ by domicile \\
\vspace{0.25cm}
giving  \textbf{Mutual Reception} [MR] by \textsl{domicile} \\
so \Mars\ receives \Saturn\ and commits his disposition to him and \Saturn\ receives it\footnotemark[1]\\
\vspace{0.25cm}
This is the strongest form of reception; the second strongest form occurs when there is mutual reception by \textsl{exaltation}. e.g. both planets are in each other's sign of exaltation. i.e. \Venus\ in \Cancer\ \Trine\ \Jupiter\ in \Pisces.

\column{0.5\textwidth}

\begin{center}
{\includegraphics[width=0.9\textwidth]{charts/01-MR-by-domicile}} \\
\end{center}

\end{columns}
\footnotetext[1]{Masha'allah [RH] p3; note that Sahl would call this a non-reception as \Saturn\ is in his Fall.}
\end{frame}
% ---------------------------------------------
\subsubsection{When there is no reception}
\begin{frame}[t]{What Constitutes Reception - Non-Reception}



\end{frame}
% ----------------------------------------------------
\begin{frame}[t]{What Constitutes Reception Continued}

Masha'allah tells us that if the matter being analyzed has to do with a King, the Exaltation rulers will have more authority over the matter than the domicile rulers.

As a general rule, reception is stronger if the planet applied to (usually the heavier planet) receives the applying planet (usually the lighter planet).

Later authors also considered reception by triplicity, term, and face but only considered it to be an \textsl{effective} reception if it involved at least two of these minor dignities i.e. received by triplicity and term or triplicity and face or term and face. Masha'allah does mention these minor dignities in his \textsl{Book of Thoughts and Intentions} but only in relation to determining which planet has the most authority; he does not use them in any of his \textsl{On Reception} examples.

\end{frame}