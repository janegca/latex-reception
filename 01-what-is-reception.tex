\subsection{What Constitutes Reception?}
\begin{frame}[t]{What Constitutes Reception?}
\textsl{Reception} happens when one planet is in the domicile or exaltation of another planet who it aspects or conjuncts.

\textbf{Examples:} The aspect (conjunction) must perfect before that of another planet intercedes.

\begin{columns}[T, onlytextwidth]
\column{0.25\textwidth}
\Mars\ 10 \Aries\ $\Rightarrow$ \Conjunction\ \Saturn\ 15 \Aries

\vspace{2cm}
\Mars\ 10 \Capricorn\ $\Rightarrow$ \Square\ \Saturn\ 20 \Aries



\column{0.01\textwidth}
\rule{.1mm}{.5\textheight}

\column{0.75\textwidth}

\Mars\ applying to the \Conjunction\ of \Saturn\ who is in \Mars's domicile (\Aries), therefore\\
\Mars\ receives \Saturn, but, \\
\Mars\ is not in a domicile (\Capricorn, \Aquarius) or Exaltation (\Libra) of \Saturn, so \\
\Saturn\ does not receive \Mars \\
\vspace{0.5cm}
\Mars\ is in \Saturn's domicile \\
\Saturn\ is in \Mars's domicile, therefore, \\
\Mars\ receives \Saturn\ and \Saturn\ receives \Mars\ and \\
there is \textbf{Mutual Reception} [MR]
\end{columns}
 \textbf{These examples also involve 
\end{frame}