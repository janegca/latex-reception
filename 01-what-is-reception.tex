\subsection{What Constitutes Reception?}
\begin{frame}[t]{What Constitutes Reception?}
\small
\begin{block}{}
\textsl{Reception} happens when one planet (A) is in the domicile or exaltation of another planet (B) whom it aspects or conjuncts. The aspect (or conjunction) from planet \textsl{A} must \textsl{perfect} (become exact)  before that of a third planet (C) intercedes by completing its own aspect or conjunction with planet \textsl{B}, or, before \textsl{B} moves into another sign.
\end{block}

\vspace{0.1cm}
\begin{columns}[T, onlytextwidth]
\column{0.25\textwidth}
\Mars\ 10 \Aries\ $\Rightarrow$ \Conjunction\ \Saturn\ 15 \Aries \\
\vspace{1.5cm}
\Mars\ 10 \Capricorn\ $\Rightarrow$ \Square\ \Saturn\ 20 \Aries

\column{0.01\textwidth}
\rule{.1mm}{.4\textheight}

\column{0.75\textwidth}
\Mars\ is applying to \Conjunction\ \Saturn\ who is in \Mars's domicile (\Aries), therefore\\
\Mars\ \textsl{receives} \Saturn, but, \\
\Mars\ is not in a domicile (\Capricorn, \Aquarius) or Exaltation (\Libra) of \Saturn, so \\
\Saturn\ \textsl{does not receive} \Mars \\
\vspace{0.1cm}
\ul
\Mars\ is in \Saturn's domicile (\Capricorn) applying to \Square\ \Saturn \\
\Saturn\ is in \Mars's domicile (\Aries), therefore, \\
\Mars\ receives \Saturn\ and \Saturn\ receives \Mars\ and \\
there is \textbf{Mutual Reception} [MR] by \textsl{domicile}
\end{columns}
\vspace{0.2cm}
 MR can happen by Exaltation if the two planets are in each other's signs of exaltation. Masha'Allah tells us that if the matter being analyzed has to do with a King, the Exaltation rulers will have more authority over the matter than the domicile rulers.
\end{frame}