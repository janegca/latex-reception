\subsection{Committing Disposition}
\begin{frame}[t]{Committing Disposition}
\small
\begin{block}{}
Any planet (A) joining another (B) from \textsl{B}'s domicile or exaltation \textsl{commits its disposition} to \textsl{B}, who, because of reception, will perfect the matter signified by \textsl{A}. A disposition may be \textsl{returned}, possibly in worse condition, if the receiving planet is retrograde, combust, or in its Fall. 
\end{block}

The language used here makes sense in that if \textsl{A} is in a domicile of \textsl{B} he is a guest in \textsl{B}'s home. Astrologically speaking, the ruler of the domicile (B) is responsible for \textsl{settling} or \textsl{disposing} of all matters within his domicile, including those of other planets lodged there. If \textsl{B} is averse (\Semisextile\ or \Quincunx) to its own sign he is blind to the needs of planets in that sign and so he cannot meet their needs unless there is some mitigating condition such as planets in signs that are:

\begin{itemize}
\item \textbf{Like-engirding:}  \Aries\ \Scorpio, \Taurus\  \Libra, \Gemini\ \Virgo, \Sagittarius\ \Pisces,  \Capricorn\ \Aquarius\ (ruled by the same planet)

\item \textbf{Equally Ascending:}  \Aries\ \Pisces, \Taurus\ \Aquarius, \Gemini\ \Capricorn, \Cancer\ \Sagittarius, \Leo\ \Scorpio, \Virgo\ \Libra\ (rise in the same number of hours)

\item \textbf{Seeing-Perceiving:} \Aries\ \Libra, \Taurus\ \Virgo, \Gemini\ \Leo, \Scorpio\ \Pisces, \Sagittarius\ \Aquarius

\item \textbf{Commanding-Obeying:} \Taurus\ \Pisces, \Gemini\ \Aquarius, \Cancer\ \Capricorn, \Leo\ \Sagittarius, \Virgo\ \Scorpio
\end{itemize}
In which cases, \Semisextile's act like a \Conjunction, a \Sextile\ has the power of a \Trine, a \Square's power is doubled, a \Quincunx\ acts like an \Opposition
\end{frame}
% ---------------------------------------------------------------------------------------------------
\begin{frame}[t]{Committing Disposition - An Example}
The nature of the aspect qualifies the nature of the request as a hard one with difficulty, delay, or labour (\Square, \Opposition); an easy one, without labour or difficulty (\Sextile, \Trine), or a familiar, or congenial one, easily done (\Conjunction).

\textbf{Example:}
\begin{columns}[T, onlytextwidth]
\column{0.25\textwidth}
\Sun\ 10 \Aries\ \Square\ \Mars\ 10 \Capricorn

\column{0.01\textwidth}
\rule{.1mm}{.27\textheight}

\column{0.75\textwidth}
the \Sun\ perfects a \Square\  to \Mars, \\
committing his disposition to \Mars, who  \\
receives the \Sun\ in his domicile (\Aries), and so \\
\textsl{accepts} the \Sun's disposition and will perfect it, but \\
due to the \Square\, \\ 
only after some anxiety and errors i.e. not easily
\end{columns}
\end{frame}
% -------------------------------------------------------------------------------------------------
\subsubsection{Committing Disposition - Another Example}
\begin{frame}[t]{Committing Disposition - Another Example}
\textbf{Example:}\footnotemark[1]
\begin{columns}[T, onlytextwidth]
\column{0.25\textwidth}
\Sun\ 1 \Libra\ $\Rightarrow$ \Opposition\ \Saturn\ 25 \Aries

\column{0.01\textwidth}
\rule{.1mm}{.20\textheight}

\column{0.75\textwidth}
the \Sun\ is in \Libra\ (the exaltation of \Saturn) applying to \Opposition\ \Saturn \\
\Saturn\ is in \Aries\ (the exaltation of the \Sun) therefore \\
there is \textbf{Mutual Reception} by \textsl{exaltation}, and \\
the \Sun\ commits his disposition to \Saturn, who accepts it
\end{columns}
\vspace{0.25cm}
According to Masha'Allah, when there is Mutual Reception, \textsl{"there is peace and the matter will be perfected"}; however, if \Saturn\ leaves \Aries\ before the \Opposition\ can perfect, \textsl{"there will be enmities, contrariness, misunderstandings, denials and \Saturn\ will not receive the \Sun"} and so will not take on the \Sun's disposition.

\footnotetext[1]{The examples used by Masha'Allah so far involve planets in their sign of Fall or Detriment which we will see (in later examples), may result in a commitment being returned or corrupted.}
\end{frame}
