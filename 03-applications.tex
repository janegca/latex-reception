\subsection{Applications and Separations}

\begin{frame}[t]{Applications and Separations}
\begin{block}{}
\textbf{Axiom}: The faster, lighter planet (A) moves to join a slower, heavier planet (B) and in so doing, \textsl{A} commits his disposition to \textsl{B} and the aspect does not leave off until the two planets are separated.
\end{block}

\begin{mdframed}[backgroundcolor=gray!5, rightmargin=2em, leftmargin=2em]
\textbf{Note:} Robert Hand says \textsl{"there is no separating orb"} and so once the two planets are no longer in a partile (exact according to degree) configuration, the matter is done; however, other authors may consider planets to be in aspect as long as they are within the moiety of both. 

The \textsl{moiety} of an aspect between two planets is equal to the addition of half of each planet's orb. For example, the \Sun\ has an orb of 15°, \Saturn\ of 9° so the moiety is 12° (7.5° + 4.5° or (15° + 9°)/2) so, for some, the pair would not be fully separated until the \Sun\ was 12° past \Saturn.
\end{mdframed}

\begin{block}{}
\textbf{Axiom:} A planet \textsl{applying to (joining)} another planet indicates \textsl{"what will be"}. 
\end{block}

\begin{block}{}
\textbf{Axiom:} A planet separating from another planet indicates \textsl{"what has passed away"}.
\end{block}

\end{frame}